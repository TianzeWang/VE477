%
% do not add anything in this part
%
\ifx\onefile\undefined

	\documentclass{article}

	%if tcolorbox and tikz are installed use next line
	\usepackage[tcbox]{projectalgo}
	\usepackage{projectalgo}
	\usepackage{enumerate}
	\usepackage{algorithm2e}
	% replace type by one of graph, math, combinatory, string, network, datastructure, ai, image
	\pbtype{graph}

	\begin{document}
\fi

%
% things can be added below
%

\def\pbname{Closure Problem} %change this, do not use any number, just the name

\section{\pbname} 

% only for overview, so short description (no more than 1-2 lines)
\begin{overview}
\item [Algorithm:] Max-weight closure finding~(algo.~\ref{problem23}) 
	% - replace nb with problem number (e.g. problem101)
	% -	must match the label of the algorithm 
	% - for more than one algo list each of them and use problem101a, problem101b, problem101c etc.
\item [Input:] A directed graph 
\item [Complexity:]$\mathcal{O}(|V|\cdot |E|^2)$
\item [Data structure compatibility:] directed graph
\item [Common applications:] open pit mining, military application, job scheduler, network flow design
\end{overview}



\begin{problem}{\pbname}
	Given a directed graph, with each vertex has some weight. Find the maximum-weight directed graph that has no edges start within the graph and end outside the graph.
\end{problem}

\subsection*{Description}
\begin{enumerate}
\item Closure: A closure is defined as a set of vertices, for a directed graph, there is no outgoing edges. And this means that there shall only be edges pointing from any node outside the closure to the nodes inside the closure.
\item Problem clarification: In this project I will focus on the maximum weight closure problem, that is to find the closure with the greatest sum of vertices' weight.
\item Problem solution: To solve the closure problem, it could be \textbf{reduced} into a maximum flow problem in the following way, which will be discussed in algo~\ref{problem23}.
\begin{enumerate}
\item Add two nodes, $s$ and $t$ to the original graph.
\item For all nodes with positive weight, add a directed edge from $s$ to the node and set the capacity of the edge to be the absolute value of node's weight.
\item For all nodes with negative weight, add a directed edge from the node to $t$ and set the capacity of the edge to be the absolute value as node's weight.
\item For all other existed edges, set their capacity to be $\infty$.
\end{enumerate}
\end{enumerate}
\newpage
% add comment in the pseudocode: \cmt{comment}
% define a function name: \SetKwFunction{shortname}{Name of the function}
% use the defined function: \shortname{$variables$}
% use the keyword ``function'': \Fn{function name}, e.g. \Fn{\shortname{$var$}}
\begin{Algorithm}[Reduction to the max flow\label{problem23}]
	% - replace nb with problem number (e.g. problem101)
	% -	must match the reference in the overview
	% - when writing more than one algo use problem101a, problem101b, problem101c etc.
	%\SetKwFunction{myfunction}{MyFunction}	
	\Input{A directed graph $G=(V,E)$}
	\Output{A new directed graph after redution}
	%	\Fn{\myfunction{$a,b$}}{
	%	}
	\BlankLine
	\SetKwFunction{rd}{reduce}
	\Fn{\rd{G}}{
	$H \leftarrow \{\}$\;
	$H.V = G.V + \{s\} + \{t\}$\;
	\For{All nodes $n_1$ in $G.V$ with $G.V.weight > 0$}{
		Add edge $<s,n_1>$ to $H.E$\;
		$<s,n_1>.capacity = n_1.weight$\;
	}
	\For{All nodes $n_2$ in $G.V$ with $G.V.weight < 0$}{
		Add edge $<n_2,t>$ to $H.E$\;
		$<s,n_1>.capacity = n_1.weight$\;
	}
	\For{All other edges in $G.E$}{
		$H.E.capacity = \infty$\;
	}
	\SetKwFunction{kw}{edmond\_karp}
	\cmt{Apply Edmond Karp algorithm on $H$, to get the max cut. Edmond Karp algorithm already discussed in class, no need to write it out here.}\\
	\kw{H}\;
	\KwRet{$H.left-\{s\}$}\
	}
\end{Algorithm}

After performing this algorithm, the returned part are the nodes within the closure.
\newpage
\subsection*{Time complexity}
For this problem, since the original problem can be reduced in a maximum flow problem within polynomial time $\mathcal{O}(|V|+|E|)$, so there are in the same complexity space, and the total time complexity is decided by the time complexity of the Edmond Karp algorithm, which is given by 
\[
	\mathcal{O}(|V|\cdot |E|^2)
\]
\subsection*{References}
% list references where to find information on the given problem
% prefer books, research articles, or internet sources that are likely to remain available over time
% as much as possible offer several options, including at least one which provide a detailed study of the problem
% if available include links to programs/code solving the problem

\begin{itemize}\itemsep .125cm
	\item Schrijver, A. (2002). "On the history of the transportation and maximum flow problems". Mathematical Programming. 91 (3): 437–445. 
	\item Ahuja, Ravindra K.; Magnanti, Thomas L.; Orlin, James B. (1993), "19.2 Maximum weight closure of a graph", Network flows, Englewood Cliffs, NJ: Prentice Hall Inc., pp. 719–724, ISBN 0-13-617549-X, MR 1205775.
	\item Picard, Jean-Claude (1976), "Maximal closure of a graph and applications to combinatorial problems", Management Science, 22 (11): 1268–1272.
\end{itemize}

\ifx\onefile\undefined
	\end{document}
\fi
