%
% do not add anything in this part
%
\ifx\onefile\undefined

	\documentclass{article}

	%if tcolorbox and tikz are installed use next line
	\usepackage[tcbox]{projectalgo}
	\usepackage{projectalgo}
	\usepackage{enumerate}
	\usepackage{algorithm2e}
	% replace type by one of graph, math, combinatory, string, network, datastructure, ai, image
	\pbtype{image}

	\begin{document}
\fi

%
% things can be added below
%

\def\pbname{Lens Distortion} %change this, do not use any number, just the name

\section{\pbname} 

% only for overview, so short description (no more than 1-2 lines)
\begin{overview}
\item [Algorithm:] Lens Distortion Correction~(algo.~\ref{problem130}) 
	% - replace nb with problem number (e.g. problem101)
	% -	must match the label of the algorithm 
	% - for more than one algo list each of them and use problem101a, problem101b, problem101c etc.
\item [Input:] An image with distortion 
\item [Complexity:]$\mathcal{O}(n)$, $n$ is the number of pixel
\item [Data structure compatibility:] Image
\item [Common applications:] Professional photography, image restoration in MR, computer vision, digital image processing.
\end{overview}



\begin{problem}{\pbname}
	Given an image with distortion, correct the distortion and output an image with the distortion fixed.
\end{problem}

\subsection*{Description}
Due to the defect in wide-angle lens or telescope lens, the image of the line from a photograph will not be straight when it is on the photograph. And this is called a distortion. The correction of such a distortion then becomes an important topic in image calibration. In this problem we will discuss about the radial distortion, which includes two types: barrel and pincushion distortion.

The radial distortion can be measured with the Brown model, \begin{equation}\label{eq:1}
	\begin{pmatrix} {x}_{res}-x_c \\ {y}_{res} - y_c\end{pmatrix} = L(r) \begin{pmatrix} x- x_c \\ y- y_c\end{pmatrix}
\end{equation}
$(x_{res},y_{res})$ is the correct position of each pixel $(x,y)$, and $L(r)$ is a function on $r = \sqrt{(x-x_c)^2+(y-y_c)^2}$, $(x_c,y_c)$ is the center of the image. 

To help restore the image, there are four stages. 
\begin{enumerate}[1.]
\item Canny method: detect an edge.\newline
In this stage, the canny method will detect the edge. The whole process can be found in the reference part.~\hyperref[canny]{[1]}
\item Hough transformation: distortion parameter estimation.\newline
For Hough transformation, we will restrict the function $L(H)$ in equation \ref{eq:1} as 
\[
	L(r) = \frac{1}{1+k_1r^2+k_2r^4+\cdots +k_ir^{2i}+\cdots}
\]	
And we will get 
\begin{equation}\label{eq:2}
	r_{res} = \frac{r}{1+k_1r^2}
\end{equation}
Note here $r_{res}$ is the distance from the center after applying the model. Since the original Hough transform will separate a line segment into several, which will not help the image restoration, an improved hough transformation is introduced. The algorithm of this improved hough transformation is described in algorithm 1.
\item Distortion parameter optimize.

In this stage, the previous parameter obtained from the hough transformation will be optimized by removing a certain number of trivial edges, which is less-visited. In the second algorithm, the main purpose it to minimize the error from previous algorithm 1.

\item Image correction.
Now we have the correction parameter, we will use the equation \ref{eq:2} to obtain \[
	(1+k_1\cdot r^2)r_{res} - r = 0
\]
Solving this equation and we can get the corrected image. 
\end{enumerate}

% add comment in the pseudocode: \cmt{comment}
% define a function name: \SetKwFunction{shortname}{Name of the function}
% use the defined function: \shortname{$variables$}
% use the keyword ``function'': \Fn{function name}, e.g. \Fn{\shortname{$var$}}
\begin{Algorithm}[Line detection with hough transformation\label{problem130}]
	% - replace nb with problem number (e.g. problem101)
	% -	must match the reference in the overview
	% - when writing more than one algo use problem101a, problem101b, problem101c etc.
	%\SetKwFunction{myfunction}{MyFunction}	
	\Input{Image coordinate, Image and maximum expected line segments $N$}
	\Output{parameter $p_0$ for normalized distortion, most used lines}
	%	\Fn{\myfunction{$a,b$}}{
	%	}
	\BlankLine
	\SetKwFunction{ld}{line\_detection}
	\Fn{\ld{Image, $N$}}{
		\For{$p = \left\{p_{min}+n \delta \right\}$, $p \leq p_{max}$}{
		\For{$(x,y)$ in image}{
		\cmt{$\alpha$ is the orientation parameter}\\
			Update $\alpha$ according to the old coordinate and old orientation parameter\;
			Mark the new $\alpha$ as $\alpha_{new}$\;
			\For{$\beta \in  [\alpha_{new}- d_\alpha, \alpha_{new}+d_\alpha]$}{
			\cmt{Update new distance}\\
				$d_l= |\cos{\beta} x_{res} + \sin{\beta} y_{res} + d |$ \;
				weight of selected line with parameter $\{dist, \beta, p\}$ + $\frac{1}{1+d_l}$ \;
			}
		
		\cmt{Sum to be all weight}\\
		}
		$sum = \sum weight(d, \alpha, p)$\;
		\If{$sum > -1$}{
		$p_0 = p$\;
		$max_{Hough} = sum$\;
		}
	}
		\KwRet{$p_0$}\;
	}

\end{Algorithm}

\begin{Algorithm}[Optimize $p_0$]
	\Input{$p_0$}
	\Output{$p_{opt}$}

	\SetKwFunction{opt}{Optimize}
	\Fn{\opt{$p_0$}}{
		$p_{opt} \leftarrow p_0$\;
		$\sigma = 1$\;
		$p = p_0 + tolerence + 1$\;
		\While{$|p_{opt}-p_0|>tolerence$}{
		\cmt{$E()$ is the expectation function}\\
			\While{$E(p)>E(p_{opt})$}{
				$\sigma = \sigma\cdot 10$\;
				$p_{new} = p_{opt} - \frac{E(p_{opt})}{E(p_{opt})+ \sigma}$\;
			}
			$\sigma = \sigma /10$\;
			$p = p_{opt}$\;
			$p_{opt} = p_{new}$\;
		}
		\KwRet{$p_{opt}$}
	}
\end{Algorithm}

\begin{Algorithm}[image restoration]
	\Input{Image and lens model}
	\Output{Perfect image without distortion}
	\SetKwFunction{imr}{image\_restoration}
	\Fn{\imr{Image, parameter $r$}}{
		\For{$(x_{res},y_{res}) \in $ Output image}{
			$r_{res} = \sqrt{(x_{res}-x_c)^2 + (y_{res}-y_c)^2}$\;
			$id = \lfloor r_{res} \rfloor$\;
			$wei = r_{res} - id$\;
			\cmt{Vector contains result from $r = \frac{1-\sqrt{1- 4k_1(r_{res})^2}}{2 k_1 r_{res}}$}
			$r = (1 - wei)\cdot Vector[id] + wei\cdot Vector[id + 1]$\;
			$(x,y) = (x_c+ r \frac{x_{res}-x_c}{r_res},y_c+ r \frac{y_{res}-y_c}{y_res})$\;
			\KwRet{The image built with $(x_{res}, y_{res})$}\;
		}
	}
\end{Algorithm}
\newpage
Something about time complexity: The time complexity of this algorithm shall be analyzed step by step. 
\begin{enumerate}
\item Edge detector: canny method. In this step, the time complexity is $\mathcal{O}(N)$, where $N$ is the number of pixels.
\item Hough transform: In this step, the time complexity is $\mathcal{O}(N_{lines})$
\item Distortion parameter optimization: In this step, the time complexity is given by $\mathcal{O}(N_{edges})$
\item Image correction: In this step, the time complexity is given by $\mathcal{O}(N_{pixels})$
\end{enumerate}
Since $N_{pixels} >> N_{lines} >> N_{edges}$, the final time complexity is given by $\mathcal{O}(N_{pixels})$
\subsection*{References}
% list references where to find information on the given problem
% prefer books, research articles, or internet sources that are likely to remain available over time
% as much as possible offer several options, including at least one which provide a detailed study of the problem
% if available include links to programs/code solving the problem

\begin{itemize}\itemsep .125cm
	\item \label{canny}J. Canny, A computational approach to edge detection, IEEE Transactions on Pattern Analysis and Machine Intelligence, 8 (1986), pp. 679–698. \url{http://dx.doi.org/10.1109/TPAMI.1986. 4767851.}
	\item M. Alem´an-Flores, L. Alvarez, L. Gomez, and D. Santana-Cedr´es, Automatic Lens Distortion Correction Using One-Parameter Division Models, Image Processing On Line, 4 (2014), pp. 327–343. \url{http://dx.doi.org/10.5201/ipol.2014.106.}
	\item L. Alvarez, L. Gomez, and J.R. Sendra, Algebraic Lens Distortion Model Estimation, Image Processing On Line, 1 (2010). \url{http://dx.doi.org/10.5201/ipol.2010.ags-alde.}
\end{itemize}

\ifx\onefile\undefined
	\end{document}
\fi
