%
% do not add anything in this part
%
\ifx\onefile\undefined

	\documentclass{article}

	%if tcolorbox and tikz are installed use next line
	\usepackage[tcbox]{projectalgo}
	\usepackage{projectalgo}
	\usepackage{algorithm2e}
	%\usepackage[linesnumbered,ruled,longend]{algorithm2e}
	% replace type by one of graph, math, combinatory, string, network, datastructure, ai, image
	\pbtype{type}

	\begin{document}
\fi

%
% things can be added below
%

\def\pbname{Clique Problem} %change this, do not use any number, just the name

\section{\pbname} 

% only for overview, so short description (no more than 1-2 lines)
\begin{overview}
\item [Algorithm:] Finding maximum clique in a given graph~(algo.~\ref{problem22}) 
	% - replace nb with problem number (e.g. problem101)
	% -	must match the label of the algorithm 
	% - for more than one algo list each of them and use problem101a, problem101b, problem101c etc.
\item [Input:] An undirected graph with a set of vertices $V$ and a set of edges $E$.
\item [Complexity:] complexity of the algorithm, e.g. $\mathcal{O}(n)$
\item [Data structure compatibility:] Undirected Graphs.
\item [Common applications:] DNA computing, sociology studies, data mining.%most common fields where this algorithm is used
\end{overview}



\begin{problem}{\pbname}
	The clique problem is to find the maximum clique in a given undirected graph. A clique is defined as all nodes are connected to each other through an edge, namely a complete subgraph.
\end{problem} 

\subsection*{Description}
%Detailed description of the problem; More detailed information on the input and complexity; more applications with details on how they relate to each other (if this is the case).
 Suppose we have an undirected graph with a set of vertices and a set of edges, we wish to find a fully connected subgraph with maximum nodes, which is known as a \textit{maximum complete subgraph}. The input shall be the undirected graph. \\
 We will introduce a basic algorithm used to solve the clique problem, called \textbf{B\&B framework}. (Branch and Bound) 
% add comment in the pseudocode: \cmt{comment}
% define a function name: \SetKwFunction{shortname}{Name of the function}
% use the defined function: \shortname{$variables$}
% use the keyword ``function'': \Fn{function name}, e.g. \Fn{\shortname{$var$}}
\begin{Algorithm}[name\label{problem22}]
	% - replace nb with problem number (e.g. problem101)
	% -	must match the reference in the overview
	% - when writing more than one algo use problem101a, problem101b, problem101c etc.
	%\SetKwFunction{myfunction}{MyFunction}	
	\Input{A graph $G$}
	\Output{Clique $C$}
	%	\Fn{\myfunction{$a,b$}}{
	%	}
	\SetKwFunction{main}{Main}
	\SetKwFunction{clique}{Clique}
	\Fn{\main{G}}{
		\cmt{C* is the final clique of our algorithm, which will be used in later algorithms}.\\
		C* $\leftarrow$ $\varnothing$\;
		\clique{C*,$\varnothing$, $V$}\;
		\Ret{C*}\;
	}
	\BlankLine
	\Fn{\clique{set C*, set C, set P}}{
		\cmt{$|C|$ means the cardinality of $C$}\\
		\If{$|C| > |C*|$}{ 
			C* $\leftarrow$ C\;
		}
		\If{$|C|+|P|>|C*|$}{
			\For{$p$ in $P$}{
				P $\leftarrow$ P $\backslash$ $\{p\}$\;
				C' $\leftarrow$ C $\cup$ $\{p\}$\;
				P $\leftarrow$ P $\cap$ vertices adjacent to $p$\;
			}
		}
	}
	\Ret

\end{Algorithm}

\subsection*{References}
% list references where to find information on the given problem
% prefer books, research articles, or internet sources that are likely to remain available over time
% as much as possible offer several options, including at least one which provide a detailed study of the problem
% if available include links to programs/code solving the problem

\begin{itemize}\itemsep .125cm
%	\item If available provide URLs, e.g.~\url{http://mywebsite.org}
	\item ~\url{http://www.cs.ecu.edu/karl/6420/spr16/Notes/NPcomplete/clique.html}
	\item Wikipedia is not acceptable if this is the unique reference
	\item Reference some books, or published articles
	\item Use reliable websites (no blog allowed) that are not likely to disappear any time soon
\end{itemize}

\ifx\onefile\undefined
	\end{document}
\fi
