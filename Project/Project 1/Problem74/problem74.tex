%
% do not add anything in this part
%
\ifx\onefile\undefined

	\documentclass{article}

	%if tcolorbox and tikz are installed use next line
	\usepackage[tcbox]{projectalgo}
	\usepackage{projectalgo}
	\usepackage{algorithm2e}
	% replace type by one of graph, math, combinatory, string, network, datastructure, ai, image
	\pbtype{string}

	\begin{document}
\fi

%
% things can be added below
%

\def\pbname{Finite state machine minimization} %change this, do not use any number, just the name

\section{\pbname} 

% only for overview, so short description (no more than 1-2 lines)
\begin{overview}
\item [Algorithm:] Finite state machine minimization~(algo.~\ref{problem74}) 
	% - replace nb with problem number (e.g. problem101)
	% -	must match the label of the algorithm 
	% - for more than one algo list each of them and use problem101a, problem101b, problem101c etc.
\item [Input:] A finite state machine
\item [Complexity:] $O(|\Sigma|\cdot |F|)$
\item [Data structure compatibility:] Finite state machine (Deterministic finite automaton)
\item [Common applications:] compilers, network protocols, theory of computation
\end{overview}



\begin{problem}{\pbname}
	Given a finite state machine, minimize the states.
\end{problem}

\subsection*{Description}
%Detailed description of the problem; More detailed information on the input and complexity; more applications with details on how they relate to each other (if this is the case).
\begin{enumerate}
\item Definition of FSM

The formal definition of a finite state machine (deterministic finite automaton) is 
\[
	M:~(Q, \Sigma, \delta, q_0, F)
\]
\begin{enumerate}
\item $Q$: finite set of states
\item $\Sigma$: finite set of input symbols
\item $\delta: Q\times \Sigma \rightarrow Q$: transition function 
\item $q_0 \in Q$: initial state
\item $F \subseteq Q$: accept state
\end{enumerate}
The automaton will accept a string $w$ if it starts at start state $q_0 $, and given each character in $w$, the transition rule will transit state to state according to $\delta$, and the final state shall halt at $F$ states.

\item Input 

The input of the algorithm shall be a well-defined finite state machine. Any illegal input should not be considered.

\item Complexity

The time complexity of this algorithm is defined by the cost of each iteration and iteration time. For the outer iteration, it will need at most $|\Sigma|$ iterations, and for the inner iteration, it will need around $|F|$ operations. Thus we will need $O(|\Sigma|\cdot |F|)$ complexity.
\newpage
\item Application

In computer science field, finite state machine can be used to treat string. It can set up several states to check whether to accept an arrival of strings or not.


In digital circuits field, finite state machine can be used to depict the behavior of a certain circuit, and it could link the combinatoric circuits with a desired function.


Finite State Machine can also be used to depict a lot of behaviors in natural science or social science. For example, it could be applied to analyze the relation in different social characters, and to analyze how an social event is carried out.


\item Detailed Algorithm

This algorithm is introduced by Hopcroft. The idea behind is partition refinement, which means partition a large set into several small sets by their behavior. 

At the very beginning, the are partitioned into two different groups, that is $\{F\}$ and $\{Q\backslash F\}$. These two groups are accepting states and rejecting states, obviously they are inequivalent.

Then it comes with the magic of this algorithm. That it separates $\{A\}$ with $\{W\backslash A\}$, where $W$ initially to be $F$, 

% add comment in the pseudocode: \cmt{comment}
% define a function name: \SetKwFunction{shortname}{Name of the function}
% use the defined function: \shortname{$variables$}
% use the keyword ``function'': \Fn{function name}, e.g. \Fn{\shortname{$var$}}
\begin{Algorithm}[FSM minimization\label{problem74}]
	% - replace nb with problem number (e.g. problem101)
	% -	must match the reference in the overview
	% - when writing more than one algo use problem101a, problem101b, problem101c etc.
	%\SetKwFunction{myfunction}{MyFunction}	
	\SetKwFunction{min}{MinimizeFSM}
	\Input{A formally defined FSM}
	\Output{Minimized FSM}
	\Fn{\min{$M:~(Q, \Sigma, \delta, q_0, F)$}}{
		set $P$ $\leftarrow$ $\{F, Q\backslash F\}$\;
		set $W$ $\leftarrow$ $\{F\}$\;
		\While{$W \neq \varnothing$ }{
			Choose $A \in W$\;
			$W \leftarrow W \backslash A$\;
			\For{$s$ in $\Sigma$}{
				X $\leftarrow$ $\{X: \delta(X \times c) \rightarrow A\}$; \cmt{In natural language, it means that X is a state that: the transition rule takes $c$ in the state $X$ will go to a state in set $A$} \\	
				\For{$Y \subset P$ s.t. $X \cap Y \neq \varnothing$ $\and$ $Y \backslash X \neq \varnothing$}{
					\If{$|X\cap Y| \leq |Y\backslash X|$}{
						W $\leftarrow$ W $\cup$ $(X \cap Y)$\;
					}
					\Else{
						W $\leftarrow$ W $\cup$ $(Y \backslash X)$\; 
					}
				}
			}
		}
	}
	%	\Fn{\myfunction{$a,b$}}{
	%	}
	\BlankLine

	\Ret

\end{Algorithm}
\end{enumerate}
\subsection*{References}
% list references where to find information on the given problem
% prefer books, research articles, or internet sources that are likely to remain available over time
% as much as possible offer several options, including at least one which provide a detailed study of the problem
% if available include links to programs/code solving the problem

\begin{itemize}\itemsep .125cm
	\item If available provide URLs, e.g.~\url{http://mywebsite.org}
	\item Wikipedia is not acceptable if this is the unique reference
	\item Reference some books, or published articles
	\item Use reliable websites (no blog allowed) that are not likely to disappear any time soon
\end{itemize}

\ifx\onefile\undefined
	\end{document}
\fi
