%
% do not add anything in this part
%
\ifx\onefile\undefined

	\documentclass{article}

	%if tcolorbox and tikz are installed use next line
	\usepackage[tcbox]{projectalgo}
	\usepackage{projectalgo}

	% replace type by one of graph, math, combinatory, string, network, datastructure, ai, image
	\pbtype{string}

	\begin{document}
\fi

%
% things can be added below
%

\def\pbname{Finite state machine minimization} %change this, do not use any number, just the name

\section{\pbname} 

% only for overview, so short description (no more than 1-2 lines)
\begin{overview}
\item [Algorithm:] Finite state machine minimization~(algo.~\ref{problem74}) 
	% - replace nb with problem number (e.g. problem101)
	% -	must match the label of the algorithm 
	% - for more than one algo list each of them and use problem101a, problem101b, problem101c etc.
\item [Input:] A finite state machine
\item [Complexity:] $\mathcal{O}(n^{2\log n})$
\item [Data structure compatibility:] Finite state machine (Deterministic finite automaton)
\item [Common applications:] compilers, network protocols, theory of computation
\end{overview}



\begin{problem}{\pbname}
	Given a finite state machine, minimize the states.
\end{problem}

\subsection*{Description}
%Detailed description of the problem; More detailed information on the input and complexity; more applications with details on how they relate to each other (if this is the case).
\begin{enumerate}
\item Definition of FSM

The formal definition of a finite state machine (deterministic finite automaton) is 
\[
	M:~(Q, \Sigma, \delta, q_0, F)
\]
\begin{enumerate}
\item $Q$: finite set of states
\item $\Sigma$: finite set of input symbols
\item $\delta: Q\times \Sigma \leftarrow Q$: transition function 
\item $q_0 \in Q$: initial state
\item $F \subseteq Q$: accept state
\end{enumerate}
The automaton will accept a string $w$ if it starts at start state $q_0 $, and given each character in $w$, the transition rule will transit state to state according to $\delta$, and the final state shall halt at $F$ states.

\item Input 

The input of the algorithm shall be a finite state machine.

\item Complexity



% add comment in the pseudocode: \cmt{comment}
% define a function name: \SetKwFunction{shortname}{Name of the function}
% use the defined function: \shortname{$variables$}
% use the keyword ``function'': \Fn{function name}, e.g. \Fn{\shortname{$var$}}
\begin{Algorithm}[FSM minimization\label{problem74}]
	% - replace nb with problem number (e.g. problem101)
	% -	must match the reference in the overview
	% - when writing more than one algo use problem101a, problem101b, problem101c etc.
	%\SetKwFunction{myfunction}{MyFunction}	
	\Input{}
	\Output{}
	%	\Fn{\myfunction{$a,b$}}{
	%	}
	\BlankLine

	\Ret

\end{Algorithm}
\end{enumerate}
\subsection*{References}
% list references where to find information on the given problem
% prefer books, research articles, or internet sources that are likely to remain available over time
% as much as possible offer several options, including at least one which provide a detailed study of the problem
% if available include links to programs/code solving the problem

\begin{itemize}\itemsep .125cm
	\item If available provide URLs, e.g.~\url{http://mywebsite.org}
	\item Wikipedia is not acceptable if this is the unique reference
	\item Reference some books, or published articles
	\item Use reliable websites (no blog allowed) that are not likely to disappear any time soon
\end{itemize}

\ifx\onefile\undefined
	\end{document}
\fi
