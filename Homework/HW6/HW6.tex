% !TEX program = xelatex
\documentclass{article}
\usepackage{geometry}
\geometry{left = 3cm, right = 3cm, top = 3cm, bottom = 3cm}
\usepackage[linesnumbered,ruled,longend]{algorithm2e}
\usepackage{amsmath}
\usepackage{amsfonts,amssymb}
\usepackage{blkarray}
\usepackage{booktabs}
\usepackage{dsfont}
\usepackage{enumerate}
\usepackage{epsf}
\usepackage{fontspec}
\usepackage{forest}
\usepackage[colorlinks=true,linkcolor=purple]{hyperref}
\usepackage{listings}
\usepackage{mathrsfs}
\usepackage{microtype}
\usepackage{multirow}
\usepackage{setspace}
\usepackage{tikz}
%\usepackage{indentfirst}
%\usepackage[usenames,dvipsnames]{xcolor}
\newfontfamily\Inputmono{Consolas}
\renewcommand\thesection{Question\ \arabic{section}}%\arabic{section}}
\renewcommand\thesubsection{(\arabic{subsection})}
\renewcommand\thesubsubsection{\arabic{subsubsection}.}
\newcommand{\qedhere}{$\hfill\ensuremath{\square}$}
\defaultfontfeatures{Mapping=tex-text,Scale=MatchLowercase}
\newcommand\mycommfont[1]{\ttfamily\textcolor{blue}{#1}}
\SetCommentSty{mycommfont}
%\setmainfont{Citadel Script}
%\setmainfont{Chalkboard}
\setmainfont{CMU Bright}
%\setmainfont{Apple Chancery}
\setmonofont{Optima}
\setsansfont{Optima}
%\renewcommand{\familydefault}{\sfdefault}
%\renewcommand{\footnotesize}{\sfdefault}
\setlength{\parskip}{0.25em}
\setlength{\parindent}{0em}

%%%%%%%%%%%Configurations for code%%%%%%%%%%%%%%%%%%%%%%%
\SetKwInOut{Input}{Input}
\SetKwInOut{Output}{Output}
\SetKwProg{Fn}{Function}{\string:}{end}
\SetKwFunction{mstnew}{MST\_New}
\SetKwFunction{tw}{TreeWeight}
\SetKwFunction{dps}{DFS}
\SetKwFunction{con}{Is\_Connected}
\SetKwFunction{hor}{Three\_Fastest\_Horses}
%%%%%%%%%%%Here is the configurations for Code%%%%%%%%%%%

%\definecolor{mygreen}{rgb}{0,0.6,0}
%\definecolor{mygray}{rgb}{0.7,0.7,0.7}
%\definecolor{mymauve}{rgb}{0.58,0,0.82}
%\definecolor{mywhite}{rgb}{1,1,1}
%\definecolor{myblack}{rgb}{0,0,0}
%\definecolor{myblue}{RGB}{27,154,154}
%\lstset{
% backgroundcolor=\color{white},
% basicstyle = \footnotesize\Inputmono,
% breakatwhitespace = false,
% breaklines = true,
% captionpos = b,
% commentstyle = \color{mygray}\bfseries,
% extendedchars = false,
% frame =shadowbox,
% framerule=0.5pt,
% frameround=tttt,
% keepspaces=true,
% keywordstyle=\color{myblue}\bfseries, % keyword style
% language = Verilog,                     % the language of code
% otherkeywords={string},
% numbers=left,
% numbersep=5pt,
% numberstyle=\tiny\color{mymauve},
% rulecolor=\color{black},
% showspaces=false,
% showstringspaces=false,
% showtabs=false,
% stepnumber=0,
% stringstyle=\color{mymauve},        % string literal style
% tabsize=2,
% title=\lstname
%}

%%%%%%%%%%%%%%%%%%%%%%%%%%%%%%%%%%%%%%%%%%%%

\begin{document}
%\setmainfont{Savoye LET}
%\setmainfont{Cormorant Upright}
\setmainfont{Cormorant Upright}
\renewcommand\arraystretch{1.5}


\thispagestyle{empty}

\begin{center}
\begin{large}
\begin{figure}[!htbp]
\centering
\includegraphics[width=0.7\textwidth]{Logo2}
\end{figure}
\hrule
\vspace*{0.25cm}
\sc{ \Large  UM--SJTU Joint Institute \vspace*{0.3em}} \\
\Large  VE477 Intro to Algorithms\\
\end{large}
\hrulefill

\vspace*{3cm}

\begin{Large}
\sc{{Homework 6}} \\
\end{Large}
\vspace*{2cm}
\begin{large}
\sc{{Wang, Tianze\\ 515370910202}} \\
\end{large}
\end{center}
\newpage
\setmainfont{Optima}
\setmonofont{Optima}
\setsansfont{Optima}
%\tableofcontents
%\newpage
\setcounter{page}{1}
\section{Perfect matching in bipartite graph}
\subsection{Identical Zero}
\begin{itemize}
\item If $G$ has no perfect matching, which means that for at least one vertex, it is not connected to any other vertex in the bipartite. In this case, the column/ row of this vertex will be all 0. Since the matrix has a row/ column to be all 0, obviously the determinant will be 0.
\item If the determinant of $G$ is 0, then $G$ can be transferred into a form that one row/ column is 0. In this case, $G$ will have a vertex do not have any neighbor, which means it is not a perfect matching.
\end{itemize}
\subsection{Algorithm}
\begin{algorithm}
\Input{A bipartite graph$G$ =  $L \cup R$}
\Output{bool: whether the graph can be perfectly matched}
\SetKwFunction{a}{PerfectMatching}
\Fn{\a{$G$}}{
	\For{All $a_{i,j}$ corresponds to the $i-th$ element in $L$ and $j-th$ element in $R$}{
		\uIf{the $i-th$ element in $L$ and $j-th$ element in $R$ are connected}{
			$a_{i,j} = X_{i,j}$\;
		}
		\Else{$a_{i,j} = 0$\;}
	}
	\uIf{$det(a) = 0$}{
		\KwRet{Can not be perfectly matched.}
	}
	\Else{
		\KwRet{Can be perfectly matched.}
	}
}
\end{algorithm}
\subsection{Complexity and Error Probability}
The complexity is $\mathcal{O}(N^2)$, where $N$ is the size of each bipartite. The error will occur when the determinate is 0 however actually each can be paired in the bipartite graph. A simple case is the matrix are all 1's and the determinant is 0, however we can definitely find the perfect matching. So the error probability is $\frac{1}{2}$ according to the proof done at this website. (\url{https://cmurandomized.wordpress.com/2011/02/09/lecture-10-polynomial-identity-testing-and-parallel-matchings/})
\subsection{Usefulness of this algorithm}
This algorithm is useful considering that it uses the adjacent matrix, which will be helpful when it's a dense graph. 
\section{Critical Thinking}
\subsection{Middle node}
Store each element while visiting, and when reaching the end, return the $\lfloor element\_num/2 \rfloor$ element from the list.
\subsection{Loop decision}
Let there be two simultaneous visiting. Each time the first one visit two nodes, and after that, the second one visit one nodes. If the first one reaches the end without meeting the second one, there is no cycle. If it meets with the first one, it will definitely be a cycle.
\section{Coupon collector disillusion}
\subsection{At least how many box?}
At least the collector should use $n$ boxes to hold the coupons.
\subsection{Not done }
\subsection{Expectation}
\begin{align}
E[X] &= E[X_1] + E[X_2]+ ... + E[X_n]\\
&= n/n + n/(n-1) + n/(n-2) + ... + n/1 \\
&\approxeq n(\log n ) + O(1) 
\end{align}
So we could conclude that 
\[
	E[X] = \Theta(n\log n)
\]
\subsection{Explain counpon collector}
The previous formula means the expected number of boxes that the collector should use to hold the coupons.
\end{document}