%!TEX program = xelatex 
\documentclass[A4paper]{article}
\usepackage{geometry}
\geometry{left = 3cm, right = 3cm, top = 3cm, bottom = 3cm}
\usepackage[linesnumbered,ruled,longend]{algorithm2e}
\usepackage{amsmath}
\usepackage{amsfonts,amssymb}
\usepackage{blkarray}
\usepackage{booktabs}
\usepackage{dsfont}
\usepackage{enumerate}
\usepackage{epsf}
\usepackage{fontspec}
\usepackage{forest}
\usepackage[colorlinks=true,linkcolor=purple]{hyperref}
\usepackage{listings}
\usepackage{mathrsfs}
\usepackage{microtype}
\usepackage{multirow}
\usepackage{setspace}
\usepackage{tikz}
%\usepackage{indentfirst}
%\usepackage[usenames,dvipsnames]{xcolor}
\newfontfamily\Inputmono{Consolas}
\renewcommand\thesection{Question\ \arabic{section}}%\arabic{section}}
\renewcommand\thesubsection{(\arabic{subsection})}
\renewcommand\thesubsubsection{\arabic{subsubsection}.}
\newcommand{\qedhere}{$\hfill\ensuremath{\square}$}
\defaultfontfeatures{Mapping=tex-text,Scale=MatchLowercase}
\newcommand\mycommfont[1]{\ttfamily\textcolor{blue}{#1}}
\SetCommentSty{mycommfont}
%\setmainfont{Citadel Script}
%\setmainfont{Chalkboard}
\setmainfont{CMU Bright}
%\setmainfont{Apple Chancery}
\setmonofont{Optima}
\setsansfont{Optima}
%\renewcommand{\familydefault}{\sfdefault}
%\renewcommand{\footnotesize}{\sfdefault}
\setlength{\parskip}{0.25em}
\setlength{\parindent}{0em}

%%%%%%%%%%%Configurations for code%%%%%%%%%%%%%%%%%%%%%%%
\SetKwInOut{Input}{Input} 
\SetKwInOut{Output}{Output} 
\SetKwProg{Fn}{Function}{\string:}{end} 
\SetKwFunction{mstnew}{MST\_New}
\SetKwFunction{tw}{TreeWeight}
\SetKwFunction{dps}{DFS}
\SetKwFunction{con}{Is\_Connected}
\SetKwFunction{hor}{Three\_Fastest\_Horses}
%%%%%%%%%%%Here is the configurations for Code%%%%%%%%%%%

%\definecolor{mygreen}{rgb}{0,0.6,0}
%\definecolor{mygray}{rgb}{0.7,0.7,0.7}
%\definecolor{mymauve}{rgb}{0.58,0,0.82}
%\definecolor{mywhite}{rgb}{1,1,1}
%\definecolor{myblack}{rgb}{0,0,0}
%\definecolor{myblue}{RGB}{27,154,154}
%\lstset{
% backgroundcolor=\color{white}, 
% basicstyle = \footnotesize\Inputmono,       
% breakatwhitespace = false,        
% breaklines = true,                 
% captionpos = b,                    
% commentstyle = \color{mygray}\bfseries,
% extendedchars = false,             
% frame =shadowbox, 
% framerule=0.5pt,
% frameround=tttt,
% keepspaces=true,
% keywordstyle=\color{myblue}\bfseries, % keyword style
% language = Verilog,                     % the language of code
% otherkeywords={string}, 
% numbers=left, 
% numbersep=5pt,
% numberstyle=\tiny\color{mymauve},
% rulecolor=\color{black},         
% showspaces=false,  
% showstringspaces=false, 
% showtabs=false,    
% stepnumber=0,         
% stringstyle=\color{mymauve},        % string literal style
% tabsize=2,          
% title=\lstname                      
%}

%%%%%%%%%%%%%%%%%%%%%%%%%%%%%%%%%%%%%%%%%%%%

\begin{document}
%\setmainfont{Savoye LET}
\setmainfont{Cormorant Upright}
\renewcommand\arraystretch{1.5}


\thispagestyle{empty}

\begin{center}
\begin{large}
\begin{figure}[!htbp]
\centering
\includegraphics[width=0.7\textwidth]{Logo2.png}
\end{figure}
\hrule
\vspace*{0.25cm}
\sc{UM--SJTU Joint Institute \vspace*{0.3em}} \\ 
VE477 Intro to Algorithms\\
\end{large}
\hrulefill

\vspace*{3cm}

\begin{Large}
\sc{{Homework 3}} \\
\end{Large}
\vspace*{2cm}
\begin{large}
\sc{{Wang, Tianze\\ 515370910202}} \\
\end{large}
\end{center}
\newpage
\setmainfont{Optima}
\setmonofont{Optima}
\setsansfont{Optima}
%\tableofcontents
%\newpage
\setcounter{page}{1}
\section{Hamilton Path}
A Hamilton Path is a path that visit each vertex in a graph exactly once.
\subsection{}
Not done yet.
\subsection{}
Not done yet.
\subsection{}
\begin{algorithm}
\caption{Hamilton Algorithm}
\SetKwFunction{hc}{Hamilton}
\Input{An undirected graph $G$}
\Output{The Hamilton Path in $G$}
\Fn{\hc{$G$}}{
	L $\leftarrow$ []\;
	S $\leftarrow$ nodes with no coming edges\;
	\If{S size >1}{
		\KwRet{No result}
	}
	\While{$S$ $\neq  \varnothing$}{
		remove $n$ from $S$\;
		Append $n$ to tail of $L$\;
		\For{node $m$ with an edge $e$ from $n$ to $m$}{
			remove $e$ from graph.
		}
	}
	\If{Graph has other edges}{
		\KwRet{No result}
	}
	\Else{
		\KwRet{L}
	}
}
\end{algorithm}


\end{document}	

\begin{algorithm}
\caption{Hamilton Algorithm}
\SetKwFunction{hc}{Hamilton}
\Input{An undirected graph $G$}
\Output{The Hamilton Path in $G$}
\Fn{\hc{$G$}}{
	Arbitrary get a permutation of vertices in any way\;
	\For{vertices in the permutation not being adjacent}{
		Mark these vertices as break points\;
	}
	\For{All break points except for one left}{
		add edge to break points, which shall be deleted later\;
	}
	main\_break\_point $\leftarrow$ the only breakpoint\;
	main\_break\_point\_list\ $\leftarrow$ []\;
	\textbf{Append} main\_break\_point \textbf{to} main\_break\_point\_list\;
	\While{There is at least one breakpoint}{
		Cut a segment from the permutation with only one break point\;
		
		\If{All break points in main\_break\_point\_list are different from newly-created break point \textbf{and}\\
			The number of new break point to be least \textbf{or} no new breakpoint}{
				Insert it in the cycle\;
				\If{no new break point}{\textbf{break}\;}
				main\_break\_point\_list $\leftarrow$ new break point\;
				\textbf{Append} main\_break\_point \textbf{to} main\_break\_point\_list\;
			}
	}
	\KwRet{The permutation, namely the Hamilton Path}
}
\end{algorithm}
\subsection{}
The complexity is mainly decided by the updating process. Now we have the number of all possible break points to be 
\[
	(N-1)^2
\]
and each iteration, it will take polynomial time ($\mathcal{O}(N)$) computation since there is no recursive call inside iteration. The maximum cost will be visiting all the possible break points, and that will cost no more than
\[
	\mathcal{O}(N^3)
\]
So the complexity is polynomial time.
\subsection{}
It belongs to cubic time complexity.





