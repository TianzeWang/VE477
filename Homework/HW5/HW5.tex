% !TEX program = xelatex
\documentclass{article}
\usepackage{geometry}
\geometry{left = 3cm, right = 3cm, top = 3cm, bottom = 3cm}
\usepackage[linesnumbered,ruled,longend]{algorithm2e}
\usepackage{amsmath}
\usepackage{amsfonts,amssymb}
\usepackage{blkarray}
\usepackage{booktabs}
\usepackage{dsfont}
\usepackage{enumerate}
\usepackage{epsf}
\usepackage{fontspec}
\usepackage{forest}
\usepackage[colorlinks=true,linkcolor=purple]{hyperref}
\usepackage{listings}
\usepackage{mathrsfs}
\usepackage{microtype}
\usepackage{multirow}
\usepackage{setspace}
\usepackage{tikz}
%\usepackage{indentfirst}
%\usepackage[usenames,dvipsnames]{xcolor}
\newfontfamily\Inputmono{Consolas}
\renewcommand\thesection{Question\ \arabic{section}}%\arabic{section}}
\renewcommand\thesubsection{(\arabic{subsection})}
\renewcommand\thesubsubsection{\arabic{subsubsection}.}
\newcommand{\qedhere}{$\hfill\ensuremath{\square}$}
\defaultfontfeatures{Mapping=tex-text,Scale=MatchLowercase}
\newcommand\mycommfont[1]{\ttfamily\textcolor{blue}{#1}}
\SetCommentSty{mycommfont}
%\setmainfont{Citadel Script}
%\setmainfont{Chalkboard}
\setmainfont{CMU Bright}
%\setmainfont{Apple Chancery}
\setmonofont{Optima}
\setsansfont{Optima}
%\renewcommand{\familydefault}{\sfdefault}
%\renewcommand{\footnotesize}{\sfdefault}
\setlength{\parskip}{0.25em}
\setlength{\parindent}{0em}

%%%%%%%%%%%Configurations for code%%%%%%%%%%%%%%%%%%%%%%%
\SetKwInOut{Input}{Input}
\SetKwInOut{Output}{Output}
\SetKwProg{Fn}{Function}{\string:}{end}
\SetKwFunction{mstnew}{MST\_New}
\SetKwFunction{tw}{TreeWeight}
\SetKwFunction{dps}{DFS}
\SetKwFunction{con}{Is\_Connected}
\SetKwFunction{hor}{Three\_Fastest\_Horses}
%%%%%%%%%%%Here is the configurations for Code%%%%%%%%%%%

%\definecolor{mygreen}{rgb}{0,0.6,0}
%\definecolor{mygray}{rgb}{0.7,0.7,0.7}
%\definecolor{mymauve}{rgb}{0.58,0,0.82}
%\definecolor{mywhite}{rgb}{1,1,1}
%\definecolor{myblack}{rgb}{0,0,0}
%\definecolor{myblue}{RGB}{27,154,154}
%\lstset{
% backgroundcolor=\color{white},
% basicstyle = \footnotesize\Inputmono,
% breakatwhitespace = false,
% breaklines = true,
% captionpos = b,
% commentstyle = \color{mygray}\bfseries,
% extendedchars = false,
% frame =shadowbox,
% framerule=0.5pt,
% frameround=tttt,
% keepspaces=true,
% keywordstyle=\color{myblue}\bfseries, % keyword style
% language = Verilog,                     % the language of code
% otherkeywords={string},
% numbers=left,
% numbersep=5pt,
% numberstyle=\tiny\color{mymauve},
% rulecolor=\color{black},
% showspaces=false,
% showstringspaces=false,
% showtabs=false,
% stepnumber=0,
% stringstyle=\color{mymauve},        % string literal style
% tabsize=2,
% title=\lstname
%}

%%%%%%%%%%%%%%%%%%%%%%%%%%%%%%%%%%%%%%%%%%%%

\begin{document}
%\setmainfont{Savoye LET}
%\setmainfont{Cormorant Upright}
\setmainfont{Cormorant Upright}
\renewcommand\arraystretch{1.5}


\thispagestyle{empty}

\begin{center}
\begin{large}
\begin{figure}[!htbp]
\centering
\includegraphics[width=0.7\textwidth]{Logo2}
\end{figure}
\hrule
\vspace*{0.25cm}
\sc{ \Large  UM--SJTU Joint Institute \vspace*{0.3em}} \\
\Large  VE477 Intro to Algorithms\\
\end{large}
\hrulefill

\vspace*{3cm}

\begin{Large}
\sc{{Homework 5}} \\
\end{Large}
\vspace*{2cm}
\begin{large}
\sc{{Wang, Tianze\\ 515370910202}} \\
\end{large}
\end{center}
\newpage
\setmainfont{Optima}
\setmonofont{Optima}
\setsansfont{Optima}
%\tableofcontents
%\newpage
\setcounter{page}{1}
\section{Partition Problem}
\subsection{Definition}
It is the task of deciding whether a given multiset $S$ of positive integers can be partitioned into two subsets $S_1$ and $S_2$ such that the sum of the numbers in $S_1$ equals the sum of the numbers in $S_2$.
\[
	S = S_1 \cup S_2 \ \land \ S_1 \cap S_2 = \varnothing\ \land \ \sum S_1 = \sum S_2
\]
\subsection{Simple Solution}
No, it is not a good decision. For example, suppose we have a set like \[
	\{1,\ 2, \ 97, \ 99 \}
\]
And we wish to partition it into two sets. Using this method will lead to a max set of $196$, however if we partition it into 
\[
	\{1,2,97\}, \{99\}
\]
This will lead to a better solution, which is $100$. 
\subsection{Recursive Algorithm}
\newpage
\section{Critical Thinking}
Here we use the idea as: The binary representative of a decimal number.
\subsection*{0-4 to 0-7}
7 can be represented as 111 in binary. Thus 0 to 7 is $000_b$ to $111_b$.
\begin{algorithm}
\Input{7}
\Output{a random integer between 0-7}
\SetKwFunction{gen}{generator}
\Fn{\gen{3}}{
	i $\leftarrow$ 3 \;
	\While{i $\neq$ 0}{
		j $\leftarrow$ get an output from black box \;
		\tcc{$<<$ means the operation of shift left}
		\uIf{j = 0 or 1}{
			$b = 0$ \;
			$a = (a << 1) + b$ \;
			$i--$ \;
		}
		\uElseIf{j = 2 or 3}{
			$b = 1$ \;
			$a = (a << 1) + b$ \;
			$i--$ \;
		}
		\Else{
			Continue\;
		}
	}
	\KwRet{a}
}
\end{algorithm}
\newpage
\subsection{0-4 to common case}
For the common case, we apply the same idea, but we need a judging condition to represent whether our result falls into the range of acceptance.
\begin{algorithm}
\Input{n}
\Output{a random integer between 0-n}
\SetKwFunction{gen}{generator}
\Fn{\gen{n}}{
	A $\leftarrow$ a binary number \;
	\While{a>n}{
	i $\leftarrow$ $\lceil \log_2 n \rceil$ \;
	\While{i $\neq$ 0}{
		j $\leftarrow$ get an output from black box \;
		\tcc{$<<$ means the operation of shift left}
		\uIf{j = 0 or 1}{
			$b = 0$ \;
			$a = (a << 1) + b$ \;
			$i--$ \;
		}
		\uElseIf{j = 2 or 3}{
			$b = 1$ \;
			$a = (a << 1) + b$ \;
			$i--$ \;
		}
		\Else{
			Continue\;
		}
	}
	}
}
\end{algorithm}
\end{document}