%!TEX program = xelatex
\documentclass[A4paper]{article}
\usepackage{geometry}
\geometry{left = 3cm, right = 3cm, top = 3cm, bottom = 3cm}
\usepackage[linesnumbered,ruled,longend]{algorithm2e}
\usepackage{amsmath}
\usepackage{amsfonts,amssymb}
\usepackage{blkarray}
\usepackage{booktabs}
\usepackage{dsfont}
\usepackage{enumerate}
\usepackage{epsf}
\usepackage{fontspec}
\usepackage{forest}
\usepackage[colorlinks=true,linkcolor=purple]{hyperref}
\usepackage{listings}
\usepackage{mathrsfs}
\usepackage{microtype}
\usepackage{multirow}
\usepackage{setspace}
\usepackage{tikz}
%\usepackage{indentfirst}
%\usepackage[usenames,dvipsnames]{xcolor}
\newfontfamily\Inputmono{Consolas}
\renewcommand\thesection{Question\ \arabic{section}}%\arabic{section}}
\renewcommand\thesubsection{(\arabic{subsection})}
\renewcommand\thesubsubsection{\arabic{subsubsection}.}
\newcommand{\qedhere}{$\hfill\ensuremath{\square}$}
\defaultfontfeatures{Mapping=tex-text,Scale=MatchLowercase}
\newcommand\mycommfont[1]{\ttfamily\textcolor{blue}{#1}}
\SetCommentSty{mycommfont}
%\setmainfont{Citadel Script}
%\setmainfont{Chalkboard}
\setmainfont{Optima}
%\setmainfont{Apple Chancery}
\setmonofont{Optima}
\setsansfont{Optima}
%\renewcommand{\familydefault}{\sfdefault}
%\renewcommand{\footnotesize}{\sfdefault}
\setlength{\parskip}{0.5em}
\setlength{\parindent}{0em}

%%%%%%%%%%%Configurations for code%%%%%%%%%%%%%%%%%%%%%%%
\SetKwInOut{Input}{Input} 
\SetKwInOut{Output}{Output} 
\SetKwProg{Fn}{Function}{\string:}{end} 
\SetKwFunction{mstnew}{MST\_New}
\SetKwFunction{tw}{TreeWeight}
\SetKwFunction{dps}{DFS}
\SetKwFunction{con}{Is\_Connected}
\SetKwFunction{hor}{Three\_Fastest\_Horses}
%%%%%%%%%%%Here is the configurations for Code%%%%%%%%%%%

%\definecolor{mygreen}{rgb}{0,0.6,0}
%\definecolor{mygray}{rgb}{0.7,0.7,0.7}
%\definecolor{mymauve}{rgb}{0.58,0,0.82}
%\definecolor{mywhite}{rgb}{1,1,1}
%\definecolor{myblack}{rgb}{0,0,0}
%\definecolor{myblue}{RGB}{27,154,154}
%\lstset{
% backgroundcolor=\color{white}, 
% basicstyle = \footnotesize\Inputmono,       
% breakatwhitespace = false,        
% breaklines = true,                 
% captionpos = b,                    
% commentstyle = \color{mygray}\bfseries,
% extendedchars = false,             
% frame =shadowbox, 
% framerule=0.5pt,
% frameround=tttt,
% keepspaces=true,
% keywordstyle=\color{myblue}\bfseries, % keyword style
% language = Verilog,                     % the language of code
% otherkeywords={string}, 
% numbers=left, 
% numbersep=5pt,
% numberstyle=\tiny\color{mymauve},
% rulecolor=\color{black},         
% showspaces=false,  
% showstringspaces=false, 
% showtabs=false,    
% stepnumber=0,         
% stringstyle=\color{mymauve},        % string literal style
% tabsize=2,          
% title=\lstname                      
%}

%%%%%%%%%%%%%%%%%%%%%%%%%%%%%%%%%%%%%%%%%%%%

\begin{document}
%\setmainfont{Savoye LET}
\setmainfont{Cormorant Upright}
\renewcommand\arraystretch{1.5}


\thispagestyle{empty}

\begin{center}
\begin{large}
\begin{figure}[!htbp]
\centering
\includegraphics[width=0.7\textwidth]{Logo2.png}
\end{figure}
\hrule
\vspace*{0.25cm}
\sc{UM--SJTU Joint Institute \vspace*{0.3em}} \\ 
VE477 Intro to Algorithms\\
\end{large}
\hrulefill

\vspace*{3cm}

\begin{Large}
\sc{{Homework 2}} \\
\end{Large}
\vspace*{2cm}
\begin{large}
\sc{{Wang, Tianze\\ 515370910202}} \\
\end{large}
\end{center}
\newpage
\setmainfont{Optima}
\setmonofont{Optima}
\setsansfont{Optima}
%\tableofcontents
%\newpage
\setcounter{page}{1}
\section{Basic complexity}
\subsection*{1. a)}
\par We first prove that $n^3 - 3n^2 -n +1 = \mathcal{O}(n^3)$. We choose $c=2$ and $n=4$, next we calculate 
\[
	c\cdot g(n)-f(n) = 2n^3-(n^3-3n^2-n+1) =n((n- \frac{3}{2})^2- \frac{13}{4})
\]
For $n>4$, obviously the former equation yields to a result greater than 0. Since we have found the $c$ and $n$ to make the condition validate, which means $n^3 - 3n^2 -n +1 = \mathcal{O}(n^3)$.
\par Next is $n^3 - 3n^2 -n +1 = \Omega (n^3)$. We choose $c=\frac{1}{2}$ and $n=7$. 
\[
	f(n)-c\cdot g(n) = (n^3-3n^2-n+1) - \frac{1}{2}n^3 = \frac{1}{2}n [(n-3)^2-11]+1
\]
For $n\geq 7$, the former equation yields to a result greater than 0, which means $n^3 - 3n^2 -n +1 = \Omega (n^3)$.
\par Since $n^3 - 3n^2 -n +1 = \mathcal{O}(n^3)$ and $n^3 - 3n^2 -n +1 = \Omega (n^3)$, we could conclude that $$n^3 - 3n^2 -n +1 = \Theta (n^3)$$
\qedhere
\subsection*{1. b)}
We set $c = 1$ and $n = 2$. We will find that when $n=2$, $2^n = n^2$, for easier comparison, we transform them into $log$ basis. which is $2\log n$ and $n \log 2$
\par then we use 
\[
	f(x) = \int f'(x)
\]
So next we need to compare $\frac{d}{dn} 2 \log n = \frac{2}{n}$ and $\log 2$.\par  Obviously, $\frac{2}{n}\leq 1$, $\forall n \geq 2$, so we have \[
	\frac{d}{dn} 2 \log n \leq \frac{d}{dn} n \log 2
\]
And then 
\[
	f(n) = 2 \log 2 + \int_{2}^n f'(n)
\]
and \[
	g(n) = 2\log 2 + \int_2^n g'n
\]
So  $\forall 	n\geq 2$, and $c =1$, \[
	f(n) \leq g(n)
\]
namely
\[
	n^2 = \mathcal{O}(2^ )
\]
\qedhere
\subsection*{2. a)}
$f(n) = \mathcal{O}(g(n))$. We choose $c = 1$ and $n = 9$. For the base case, namely $f(9)$ and $g(9)$, $f(n) \leq g(n)$. And we apply the same methods as 1.b), since $f'(n)<g'(n)$, $\forall n \geq 9$, we could conclude that 
\[
f(n) = \mathcal{O}(g(n))
\] 
\subsection*{3. a)}
Not exist.
\subsection*{3. b)}
$f(n) = n$, $g(n) = 10$ 
\subsection*{4}
When n is approaching $\infty$,
\[
	f_4(n) > f_1(n) > f_3(n) > f_2(n)
\]
It is easy to obtain the order of $f_2$ and $f_3$, \[
	\frac{f_3}{f_2} = \frac{\sqrt{n}}{\sqrt{\log n}} >1 \Rightarrow  f_3 > f_2
\]
\par Next we need to compare $f_3$ and $f_1$. After observing the form of $f_3$ and $f_1$, we divide them into pairs, namely $p_i= \sqrt{i}+\sqrt{n+1-i}$ for $f_1$ and $q = 2 \sqrt{n}$ for $f_3$.
\par Note that $f_1 = \sum_{i=1}^{n/2} p_i$ and $f_3 = \sum_{i=1}^{n/2} q $.
Then we calculate $p_1^2-q^2$,
\[
	p_1^2-q^2 = n+ 2\sqrt{n}+1 - 4 \log n > n - 4\log n
\]
when $n\geq 9$, we will have $f_1^2-f_3^2>0$. Similarly, we can derive that for every pair, $p_i > q$. And this tells $f_1>f_3$.
\par $f_4>f_1$ is also obvious. That 
\[
	f_4 > n \sqrt{n} > \underbrace{\sqrt{n} + \sqrt{n} + \cdots + \sqrt{n}}_{\text{totally n items}} > 1 + \sqrt{2} + \cdots + \sqrt{n} = f_1
\]
\qedhere
\section{Master Theorem}
\subsection*{1 a)}
\begin{center}
\begin{forest}
[$f(n)$,draw
	[$f(n/b)$,draw
		[$f(n/b^2)$,draw
			[$\cdots$]
			[$\cdots$]
			[$\cdots$]
		]
		[$\cdots$]
		[$f(n/b^2)$,draw
			[$\cdots$]
			[$\cdots$]
			[$\cdots$]
		]
	]
	[$\cdots$]
	[$\cdots$]
	[$\cdots$]
	[$\cdots$]
	[$\cdots$]
	[$f(n/b)$,draw
		[$f(n/b^2)$,draw
			[$\cdots$]
			[$\cdots$]
			[$\cdots$]
		]
		[$\cdots$]
		[$f(n/b^2)$,draw
			[$\cdots$]
			[$\cdots$]
			[$\cdots$]
		]
	]
]
\end{forest}
\end{center}
where each node has $b$ number of child nodes.
\subsection*{1 b)}
\begin{enumerate}[i)]
\item The depth of the tree is $\log_{b}n$
\item The leaves are $\displaystyle a^{depth} = a^{\log_{b}n}$
\end{enumerate}
\end{document}










