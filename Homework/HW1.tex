%!TEX program = xelatex
\documentclass{article}
\textwidth 155mm \textheight 220mm \oddsidemargin 5mm
\evensidemargin 5mm \topmargin -10mm
\usepackage{amsmath}
\usepackage{amsfonts,amssymb}
\usepackage{fontspec}
\usepackage{microtype}
\usepackage{setspace}
\usepackage{multirow}
\usepackage{blkarray}
\usepackage{tikz}
\usepackage{dsfont}
\usepackage{booktabs}
\usepackage{enumerate}
%\usepackage{indentfirst}
\usepackage{mathrsfs}
\usepackage{listings}
%\usepackage[usenames,dvipsnames]{xcolor}
\usepackage{epsf}
\usepackage[linesnumbered,boxed]{algorithm2e}
\usepackage[colorlinks,linkcolor=purple]{hyperref}
\newfontfamily\Inputmono{Consolas}
\renewcommand\thesection{Question\ \arabic{section}}%\arabic{section}}
\renewcommand\thesubsection{(\alph{subsection})}
\renewcommand\thesubsubsection{\arabic{subsubsection}.}
\defaultfontfeatures{Mapping=tex-text,Scale=MatchLowercase}
%\setmainfont{Citadel Script}
%\setmainfont{Chalkboard}
\setmainfont{Optima}
%\setmainfont{Apple Chancery}
\setmonofont{Optima}
\setsansfont{Optima}
%\renewcommand{\familydefault}{\sfdefault}
%\renewcommand{\footnotesize}{\sfdefault}
\setlength{\parskip}{0.2em}
\setlength{\parindent}{0em}
%%%%%%%%%%%Here is the configurations for Code%%%%%%%%%%%

\definecolor{mygreen}{rgb}{0,0.6,0}
\definecolor{mygray}{rgb}{0.7,0.7,0.7}
\definecolor{mymauve}{rgb}{0.58,0,0.82}
\definecolor{mywhite}{rgb}{1,1,1}
\definecolor{myblack}{rgb}{0,0,0}
\definecolor{myblue}{RGB}{27,154,154}
\lstset{
 backgroundcolor=\color{white}, 
 basicstyle = \footnotesize\Inputmono,       
 breakatwhitespace = false,        
 breaklines = true,                 
 captionpos = b,                    
 commentstyle = \color{mygray}\bfseries,
 extendedchars = false,             
 frame =shadowbox, 
 framerule=0.5pt,
 frameround=tttt,
 keepspaces=true,
 keywordstyle=\color{myblue}\bfseries, % keyword style
 language = Verilog,                     % the language of code
 otherkeywords={string}, 
 numbers=left, 
 numbersep=5pt,
 numberstyle=\tiny\color{mymauve},
 rulecolor=\color{black},         
 showspaces=false,  
 showstringspaces=false, 
 showtabs=false,    
 stepnumber=0,         
 stringstyle=\color{mymauve},        % string literal style
 tabsize=2,          
 title=\lstname                      
}

%%%%%%%%%%%%%%%%%%%%%%%%%%%%%%%%%%%%%%%%%%%%

\begin{document}
%\setmainfont{Savoye LET}
\setmainfont{Cormorant Upright}
\renewcommand\arraystretch{1.5}


\thispagestyle{empty}

\begin{center}
\begin{large}
\begin{figure}[!htbp]
\centering
\includegraphics[width=0.7\textwidth]{Logo2.png}
\end{figure}
\hrule
\vspace*{0.25cm}
\sc{UM--SJTU Joint Institute \vspace*{0.3em}} \\ 
VE477 Intro to Algorithms\\
\end{large}
\hrulefill

\vspace*{3cm}

\begin{Large}
\sc{{Homework 1}} \\
\end{Large}
\vspace*{2cm}
\begin{large}
\sc{{Wang, Tianze, 515370910202}} \\
\end{large}
\end{center}
\newpage
\setmainfont{Optima}
\setmonofont{Optima}
\setsansfont{Optima}
%\tableofcontents
%\newpage
\setcounter{page}{1}
\section{Hash Tables}
\subsection{}
The condition \textit{exactly k keys hash to a same slot} means \textit{exactly k keys hash to a same slot and the rest $(n-k)$ keys hash to the other slots}.
Then the probability is calculated as \[
	P = \frac{\text{Combs satisfying the condition}}{\text{All Combs}} = \frac{\binom{n}{k}\cdot 1^{n-k	}\cdot (n-1)^{n-k}}{n^n} = \binom{1}{n}^k \left(1- \frac{1}{n} \right)^{n-k} \binom{n}{k}
\]	
The numerator means \textit{choosing k numbers from n which only belongs to one slot, and the rest has totally $(n-1)$ spaces to go.} And the denominator means \textit{All combinations for n numbers.}
\subsection{}
The strongest requirement for \textit{most keys to have exactly 1 key hash to a place.} And in this case, all entries are filled by one element. So the probability is then $nP_k$.
\subsection{}

\end{document}